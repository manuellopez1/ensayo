\documentclass[12pt]{article}
\usepackage[spanish]{babel}
\usepackage{amsmath}
\usepackage{graphicx}

\begin{document}

\begin{center}
\bf{\sc\Large Gödel y Turing: El inicio de una revolución}\\
\end{center}
\vspace{120pt}
\begin{center}
\bf{\sc\Large Manuel Alejandro Lopez Loaiza }\\
\end{center}
\vspace{200pt}
\begin{center}
\bf{\sc\Large Universidad de Antioquia}
\end{center}
\begin{center}
\bf{\sc\Large facultad de ingeniería}\\
\end{center}\
\begin{center}
\bf{\sc\Large ingeniería electrónica}
\end{center}
\begin{center}
\bf{\sc\Large Medellin}\\
\end{center}\
\begin{center}
\bf{\sc\Large 2020}\\
\end{center}\



\newpage
\vspace{10PT}
\large
 Los comienzos de un nuevo siglo traen consigo una serie de acontecimientos y sucesos que ponen en cuestionamiento muchos de los mandatos que, por orden, no se deben ni se tienen que refutar. Entrando el siglo XX, en el campo de las ciencias exactas se produjo un hecho que tuvo por nombre “Crisis de los fundamentos”; fue controversial ya que se tenia erróneamente, la idea de que las matemáticas eran infalibles.

\vspace{10PT}
Esta crisis llevó a la publicación del articulo “Teoremas de la Incompletitud” propuesta por el famoso lógico Kurt Gödel; y a su vez, basándose en su argumento, por el joven matemático Alan Turing por medio de su maquina universal, dispositivo que sirvió como base teórica para la solución de algoritmos y además para lo que hoy se conoce como computación.

\vspace{10PT}
En el año 1900, durante el Congreso Internacional de Matemáticos celebrado en París, el matemático David Hilbert presentó una lista de problemas matemáticos con 23 puntos, de los cuales la gran mayoría han sido solucionados y solo 7 se mantienen como incógnitas. 

\vspace{10PT}
Fue a finales de los años 20 que Kurt Gödel trató de dar respuesta a lo expuesto por Hilbert, específicamente al punto número 2, el cual buscaba demostrar que los axiomas de la aritmética tienen consistencia lógica y que entre estos no hay alguna contradicción. Gödel propuso entonces el ya antes mencionado “Teorema de la Incompletitud”, en el cual afirma que ningún grupo de axiomas que no se contradigan entre ellos y sean estables es incompleto y también que un sistema de axiomas no puede demostrar su propia consistencia, pues siempre se debe agregar otro axioma, validando así que la aritmética no es un sistema finito y cerrado como se pensaba que se podía formalizar, abriendo así un nuevo espectro en las matemáticas.

\vspace{10PT}
Recogiendo lo anteriormente descrito, hacia el año 1936 el matemático ingles Alan Turing intentó, también encontrar un algoritmo que definiera cuales fórmulas lógicas eran o no eran teoremas; además también buscaba un método de decisión efectiva para terminar con la indecisión que en ocasiones se tornaba alrededor de algunos problemas. Turing llegó a la conclusión de que tal algoritmo era imposible de construir, es decir que no había algoritmo para decidir si algunos problemas aritméticos eran ciertos o falsos, validando así el Teorema de Gödel.

\vspace{10PT}
Después de lo descubierto Turing se dio a la tarea de crear una máquina que pudiera computar lo humanamente computable y fue así como nació la máquina de Turing la cual por definición es “un modelo matemático consistente en un autómata que es capaz de implementar cualquier problema matemático expresado a través de un algoritmo”\cite{thoth38_2016}. La cual esta formado por una cinta de papel infinita dividida en casillas, una cabeza que puede leer y sobrescribir símbolos en las casillas —y además mover la cinta hacia la derecha o la izquierda— y una serie de instrucciones y estados de partida, que configuraban el “programa” de la máquina.

\vspace{10PT}
Esta máquina contiene los principios fundamentales de la computadora digital, ya que de una forma simple la maquina solo se dedica a resolver algoritmos y tareas específicas que el humano ingrese, si lo ponemos en términos de computación moderna podremos decir que un computador son una serie de máquinas de Turing las cuales realizan procesos previamente codificados por un humano. 

\vspace{10PT}
En su epitafio David Hilbert tiene grabada la frase "Wir müssen wissen, wir werden wissen" que traduce "Debemos saber, sabremos" haciendo referencia a su creencia de lo infalible que tenia que llegar a ser la matemática. A pesar de que la “ciencia exacta” no era tan exacta después de todo; las teorías llevadas por Gödel y Turing generaron un nuevo mundo de posibilidades no solo para la ciencia, si no también para el ciudadano del común. El computador, los smartphones, los relojes inteligentes etc. son el resultado de un proceso arduo que empezó con un pequeño cuestionamiento y que, tristemente; Hilbert, Gödel y especialmente Turing no pudieron ver materializados pero que hoy en día son parte de nuestro diario vivir. No es erróneo, ni irreverente cuestionar cada ámbito del diario vivir, pues es gracias a esto que se dan los grandes avances. Gödel y Turing enseñan que no importa cuán ortodoxo y arcaico sea un tema, siempre se va a poder cuestionar y, son precisamente estas discusiones las que llevan a la evolución de la humanidad. De las grandes incógnitas es donde salen los descubrimientos mas relevantes de la sociedad.


 

\cite{gazeta_2019}
\cite{el año de turing_2013}
\cite{openmind_2018}


\bibliographystyle{apalike}
\bibliography{bibliography.bib}




\end{document}
